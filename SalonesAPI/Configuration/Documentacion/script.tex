--// use este
Scaffold-DbContext "Server=WIN-DESARROLLO\DEVSQLSERVER;Database=pruebas;user=simplexwebuser;Password=Ic3b3rg2021**;" Microsoft.EntityFrameworkCore.SqlServer -OutputDir ModelsDB -ContextDir ModelsDB -Context Context -f
Scaffold-DbContext "Server=WIN-DESARROLLO\DEVSQLSERVER;Database=pruebas;user=simplexwebuser;Password=Ic3b3rg2021**;trustServerCertificate=true;" Microsoft.EntityFrameworkCore.SqlServer -OutputDir ModelsDB -ContextDir ModelsDB -Context Context -UseDatabaseNames -NoPluralize -force
//no usar
--Scaffold-DbContext "Server=WIN-DESARROLLO\DEVSQLSERVER;Database=pruebas;user=simplexwebuser;Password=Ic3b3rg2021**;" Microsoft.EntityFrameworkCore.SqlServer -OutputDir ModelsDB -ContextDir ModelsDB -Context Context -f
--Scaffold-DbContext "Server=WIN-DESARROLLO\DEVSQLSERVER;Database=pruebas;user=simplexwebuser;Password=Ic3b3rg2021**;trustServerCertificate=true;" Microsoft.EntityFrameworkCore.SqlServer -OutputDir ModelsDB -ContextDir ModelsDB -Context Context -UseDatabaseNames -NoPluralize -force

--//inicio
Bricelam.EntityFrameworkCore.Pluralizer
Microsoft.EntityFrameworkCore.Design - version 5
Microsoft.EntityFrameworkCore - version 5
Microsoft.EntityFrameworkCore.SqlServer -version 5
Microsoft.EntityFrameworkCore.Tools - version 5
Microsoft.EntityFrameworkCore.Relational -version 5
--drop table Personas
CREATE TABLE Personas
(
	id int not null primary key identity(1,1),
	identificacion varchar(15) not null,
	primerNombre varchar(100) not null,
    segundoNombre  varchar(100)  null,-- no puede ser obligatorio
    primerApellido varchar(100) not null,
    segundoApellido varchar(100) null,-- no puede ser obligatorio
    telefono varchar(10) not null,
    correo nvarchar(100) not null,
    --idDepartamento int not null,
    idCiudad int not null,
    edad int not null,
    fechaCreacion datetime not null default GETDATE(),
	fechaActualizacion datetime null default GETDATE(),-- no puede ser obligatorio
    estado bit not null default 1
)
alter table Personas alter column segundoNombre  varchar(100)  null
alter table Personas alter column segundoApellido  varchar(100)  null
--drop table Salones
CREATE TABLE Salones
(
  id int not null primary key identity(1,1),
  idPersonaCliente int not null,
  fechaEvento datetime not null default GETDATE(),
  cantidadPersona int not null,
  idMotivo int not null,
  observacion nvarchar(max)  not null,
  estado bit not null default 1
)
alter table Salones add fechaCreacion datetime not null default GETDATE() 
alter table Salones add fechaActualizacion datetime null -- no puede ser obligatorio
--DROP TABLE Paises
CREATE TABLE [dbo].[Paises](
	[paisCodigo] [char](3) NOT NULL CONSTRAINT [DF__Pais__PaisCodigo__00200768]  DEFAULT (''),
	[paisNombre] [char](52) NOT NULL CONSTRAINT [DF__Pais__PaisNombre__01142BA1]  DEFAULT (''),
	[paisContinente] [varchar](50) NOT NULL CONSTRAINT [DF__Pais__PaisContin__02084FDA]  DEFAULT ('America del Sur'),
	[id] [int] NOT NULL,
	[codigoDian] [int] NULL,
 CONSTRAINT [PK_Pais] PRIMARY KEY NONCLUSTERED 
(
	[Id] ASC
)WITH (PAD_INDEX = OFF, STATISTICS_NORECOMPUTE = OFF, IGNORE_DUP_KEY = OFF, ALLOW_ROW_LOCKS = ON, ALLOW_PAGE_LOCKS = ON) ON [PRIMARY],
 CONSTRAINT [UK_Pais] UNIQUE NONCLUSTERED 
(
	[PaisCodigo] ASC
)WITH (PAD_INDEX = OFF, STATISTICS_NORECOMPUTE = OFF, IGNORE_DUP_KEY = OFF, ALLOW_ROW_LOCKS = ON, ALLOW_PAGE_LOCKS = ON) ON [PRIMARY]
) ON [PRIMARY]

GO

--DROP TABLE Departamentos
CREATE TABLE [dbo].[Departamentos](
	[Id] [int] NOT NULL,
	[IdPais] [int] NOT NULL,
	[DistritoDepartamento] [varchar](50) NULL,
	[CodigoDian] [int] NULL,
PRIMARY KEY CLUSTERED 
(
	[Id] ASC
)WITH (PAD_INDEX = OFF, STATISTICS_NORECOMPUTE = OFF, IGNORE_DUP_KEY = OFF, ALLOW_ROW_LOCKS = ON, ALLOW_PAGE_LOCKS = ON) ON [PRIMARY]
) ON [PRIMARY]


ALTER TABLE [dbo].[Departamentos]  WITH CHECK ADD  CONSTRAINT [DepartamentosPais] FOREIGN KEY([IdPais])
REFERENCES [dbo].[Paises] ([Id])
GO

ALTER TABLE [dbo].[Departamentos] CHECK CONSTRAINT [DepartamentosPais]
GO

--DROP TABLE Ciudades
CREATE TABLE [dbo].[Ciudades](
	[id] [int] NOT NULL,
	[ciudadNombre] [char](35) NOT NULL DEFAULT (''),
	[idDepartamento] [int] NULL,
	[codigoDian] [int] NULL,
PRIMARY KEY CLUSTERED 
(
	[Id] ASC
)WITH (PAD_INDEX = OFF, STATISTICS_NORECOMPUTE = OFF, IGNORE_DUP_KEY = OFF, ALLOW_ROW_LOCKS = ON, ALLOW_PAGE_LOCKS = ON) ON [PRIMARY]
) ON [PRIMARY]

GO

ALTER TABLE [dbo].[Ciudades]  WITH CHECK ADD  CONSTRAINT [CiudadesDepartamentos] FOREIGN KEY([IdDepartamento])
REFERENCES [dbo].[Departamentos] ([Id])
GO

ALTER TABLE [dbo].[Ciudades] CHECK CONSTRAINT [CiudadesDepartamentos]
GO

CREATE TABLE Edades
(
	id int not null primary key identity(1,1),
	edad int not null,
	Descripcion varchar(20)not null,
    fechaCreacion datetime not null default GETDATE(),
	fechaActualizacion datetime null default GETDATE(),
    estado bit not null default 1
)

ALTER TABLE Personas ADD  CONSTRAINT [UQ_PersonasIdentida] UNIQUE NONCLUSTERED(identificacion)
ALTER TABLE Personas  WITH noCHECK ADD  CONSTRAINT [FK_Personas] FOREIGN KEY(idCiudad)REFERENCES Ciudades (id)

ALTER TABLE Salones  WITH noCHECK ADD  CONSTRAINT [FK_Salones] FOREIGN KEY(idPersonaCliente)REFERENCES Personas (id)


CREATE TABLE Motivos
(
	id int not null primary key identity(1,1),
	edad int not null,
	motivo nvarchar(60)not null,
    fechaCreacion datetime not null default GETDATE(),
	fechaActualizacion datetime null,
    estado bit not null default 1
)


ALTER TABLE Salones  WITH noCHECK ADD  CONSTRAINT [FK_Motivos] FOREIGN KEY(idMotivo)REFERENCES Motivos (id)
--drop view ViewSolicitudesPorFecha
CREATE VIEW  ViewSolicitudesPorFecha as
select s.id
,s.fechaEvento
,fechaEventoTex =CONVERT(varchar,FORMAT(s.fechaEvento, 'yyyy/MM/dd HH:mm','en-US'))
,s.estado
,p.primerNombre
,p.segundoNombre
,p.primerApellido
,p.segundoApellido
,p.correo
,p.edad
,p.identificacion
,p.telefono
,s.cantidadPersona
,s.observacion
,m.motivo
,c.ciudadNombre
,d.DistritoDepartamento
,pp.paisNombre
from Salones s
join Personas p on p.id=s.idPersonaCliente
join Ciudades c on c.id =p.idCiudad
join Departamentos d on d.Id=c.idDepartamento
join Paises pp on pp.id=d.IdPais
join Motivos m on m.id=s.idMotivo


--select * from Salones

create procedure SpDeleteReserva
@id int 
as
begin 
--declare @id int =1--prueba 
if EXISTS(select COUNT(id)cantidad from Salones where id=@id)
	begin
	delete Salones where id=@id 
	select respuesta=@@ROWCOUNT; 
	end
else 
	select respuesta=@@ROWCOUNT--respuesta= 'No hay registros!';
end